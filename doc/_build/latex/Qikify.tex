% Generated by Sphinx.
\def\sphinxdocclass{report}
\documentclass[letterpaper,10pt,english]{sphinxmanual}
\usepackage[utf8]{inputenc}
\DeclareUnicodeCharacter{00A0}{\nobreakspace}
\usepackage[T1]{fontenc}
\usepackage{babel}
\usepackage{times}
\usepackage[Bjarne]{fncychap}
\usepackage{longtable}
\usepackage{sphinx}
\usepackage{multirow}


\title{Qikify Documentation}
\date{March 27, 2012}
\release{0.2.0}
\author{Nate Kupp}
\newcommand{\sphinxlogo}{}
\renewcommand{\releasename}{Release}
\makeindex

\makeatletter
\def\PYG@reset{\let\PYG@it=\relax \let\PYG@bf=\relax%
    \let\PYG@ul=\relax \let\PYG@tc=\relax%
    \let\PYG@bc=\relax \let\PYG@ff=\relax}
\def\PYG@tok#1{\csname PYG@tok@#1\endcsname}
\def\PYG@toks#1+{\ifx\relax#1\empty\else%
    \PYG@tok{#1}\expandafter\PYG@toks\fi}
\def\PYG@do#1{\PYG@bc{\PYG@tc{\PYG@ul{%
    \PYG@it{\PYG@bf{\PYG@ff{#1}}}}}}}
\def\PYG#1#2{\PYG@reset\PYG@toks#1+\relax+\PYG@do{#2}}

\def\PYG@tok@gd{\def\PYG@tc##1{\textcolor[rgb]{0.63,0.00,0.00}{##1}}}
\def\PYG@tok@gu{\let\PYG@bf=\textbf\def\PYG@tc##1{\textcolor[rgb]{0.50,0.00,0.50}{##1}}}
\def\PYG@tok@gt{\def\PYG@tc##1{\textcolor[rgb]{0.00,0.25,0.82}{##1}}}
\def\PYG@tok@gs{\let\PYG@bf=\textbf}
\def\PYG@tok@gr{\def\PYG@tc##1{\textcolor[rgb]{1.00,0.00,0.00}{##1}}}
\def\PYG@tok@cm{\let\PYG@it=\textit\def\PYG@tc##1{\textcolor[rgb]{0.25,0.50,0.56}{##1}}}
\def\PYG@tok@vg{\def\PYG@tc##1{\textcolor[rgb]{0.73,0.38,0.84}{##1}}}
\def\PYG@tok@m{\def\PYG@tc##1{\textcolor[rgb]{0.13,0.50,0.31}{##1}}}
\def\PYG@tok@mh{\def\PYG@tc##1{\textcolor[rgb]{0.13,0.50,0.31}{##1}}}
\def\PYG@tok@cs{\def\PYG@tc##1{\textcolor[rgb]{0.25,0.50,0.56}{##1}}\def\PYG@bc##1{\colorbox[rgb]{1.00,0.94,0.94}{##1}}}
\def\PYG@tok@ge{\let\PYG@it=\textit}
\def\PYG@tok@vc{\def\PYG@tc##1{\textcolor[rgb]{0.73,0.38,0.84}{##1}}}
\def\PYG@tok@il{\def\PYG@tc##1{\textcolor[rgb]{0.13,0.50,0.31}{##1}}}
\def\PYG@tok@go{\def\PYG@tc##1{\textcolor[rgb]{0.19,0.19,0.19}{##1}}}
\def\PYG@tok@cp{\def\PYG@tc##1{\textcolor[rgb]{0.00,0.44,0.13}{##1}}}
\def\PYG@tok@gi{\def\PYG@tc##1{\textcolor[rgb]{0.00,0.63,0.00}{##1}}}
\def\PYG@tok@gh{\let\PYG@bf=\textbf\def\PYG@tc##1{\textcolor[rgb]{0.00,0.00,0.50}{##1}}}
\def\PYG@tok@ni{\let\PYG@bf=\textbf\def\PYG@tc##1{\textcolor[rgb]{0.84,0.33,0.22}{##1}}}
\def\PYG@tok@nl{\let\PYG@bf=\textbf\def\PYG@tc##1{\textcolor[rgb]{0.00,0.13,0.44}{##1}}}
\def\PYG@tok@nn{\let\PYG@bf=\textbf\def\PYG@tc##1{\textcolor[rgb]{0.05,0.52,0.71}{##1}}}
\def\PYG@tok@no{\def\PYG@tc##1{\textcolor[rgb]{0.38,0.68,0.84}{##1}}}
\def\PYG@tok@na{\def\PYG@tc##1{\textcolor[rgb]{0.25,0.44,0.63}{##1}}}
\def\PYG@tok@nb{\def\PYG@tc##1{\textcolor[rgb]{0.00,0.44,0.13}{##1}}}
\def\PYG@tok@nc{\let\PYG@bf=\textbf\def\PYG@tc##1{\textcolor[rgb]{0.05,0.52,0.71}{##1}}}
\def\PYG@tok@nd{\let\PYG@bf=\textbf\def\PYG@tc##1{\textcolor[rgb]{0.33,0.33,0.33}{##1}}}
\def\PYG@tok@ne{\def\PYG@tc##1{\textcolor[rgb]{0.00,0.44,0.13}{##1}}}
\def\PYG@tok@nf{\def\PYG@tc##1{\textcolor[rgb]{0.02,0.16,0.49}{##1}}}
\def\PYG@tok@si{\let\PYG@it=\textit\def\PYG@tc##1{\textcolor[rgb]{0.44,0.63,0.82}{##1}}}
\def\PYG@tok@s2{\def\PYG@tc##1{\textcolor[rgb]{0.25,0.44,0.63}{##1}}}
\def\PYG@tok@vi{\def\PYG@tc##1{\textcolor[rgb]{0.73,0.38,0.84}{##1}}}
\def\PYG@tok@nt{\let\PYG@bf=\textbf\def\PYG@tc##1{\textcolor[rgb]{0.02,0.16,0.45}{##1}}}
\def\PYG@tok@nv{\def\PYG@tc##1{\textcolor[rgb]{0.73,0.38,0.84}{##1}}}
\def\PYG@tok@s1{\def\PYG@tc##1{\textcolor[rgb]{0.25,0.44,0.63}{##1}}}
\def\PYG@tok@gp{\let\PYG@bf=\textbf\def\PYG@tc##1{\textcolor[rgb]{0.78,0.36,0.04}{##1}}}
\def\PYG@tok@sh{\def\PYG@tc##1{\textcolor[rgb]{0.25,0.44,0.63}{##1}}}
\def\PYG@tok@ow{\let\PYG@bf=\textbf\def\PYG@tc##1{\textcolor[rgb]{0.00,0.44,0.13}{##1}}}
\def\PYG@tok@sx{\def\PYG@tc##1{\textcolor[rgb]{0.78,0.36,0.04}{##1}}}
\def\PYG@tok@bp{\def\PYG@tc##1{\textcolor[rgb]{0.00,0.44,0.13}{##1}}}
\def\PYG@tok@c1{\let\PYG@it=\textit\def\PYG@tc##1{\textcolor[rgb]{0.25,0.50,0.56}{##1}}}
\def\PYG@tok@kc{\let\PYG@bf=\textbf\def\PYG@tc##1{\textcolor[rgb]{0.00,0.44,0.13}{##1}}}
\def\PYG@tok@c{\let\PYG@it=\textit\def\PYG@tc##1{\textcolor[rgb]{0.25,0.50,0.56}{##1}}}
\def\PYG@tok@mf{\def\PYG@tc##1{\textcolor[rgb]{0.13,0.50,0.31}{##1}}}
\def\PYG@tok@err{\def\PYG@bc##1{\fcolorbox[rgb]{1.00,0.00,0.00}{1,1,1}{##1}}}
\def\PYG@tok@kd{\let\PYG@bf=\textbf\def\PYG@tc##1{\textcolor[rgb]{0.00,0.44,0.13}{##1}}}
\def\PYG@tok@ss{\def\PYG@tc##1{\textcolor[rgb]{0.32,0.47,0.09}{##1}}}
\def\PYG@tok@sr{\def\PYG@tc##1{\textcolor[rgb]{0.14,0.33,0.53}{##1}}}
\def\PYG@tok@mo{\def\PYG@tc##1{\textcolor[rgb]{0.13,0.50,0.31}{##1}}}
\def\PYG@tok@mi{\def\PYG@tc##1{\textcolor[rgb]{0.13,0.50,0.31}{##1}}}
\def\PYG@tok@kn{\let\PYG@bf=\textbf\def\PYG@tc##1{\textcolor[rgb]{0.00,0.44,0.13}{##1}}}
\def\PYG@tok@o{\def\PYG@tc##1{\textcolor[rgb]{0.40,0.40,0.40}{##1}}}
\def\PYG@tok@kr{\let\PYG@bf=\textbf\def\PYG@tc##1{\textcolor[rgb]{0.00,0.44,0.13}{##1}}}
\def\PYG@tok@s{\def\PYG@tc##1{\textcolor[rgb]{0.25,0.44,0.63}{##1}}}
\def\PYG@tok@kp{\def\PYG@tc##1{\textcolor[rgb]{0.00,0.44,0.13}{##1}}}
\def\PYG@tok@w{\def\PYG@tc##1{\textcolor[rgb]{0.73,0.73,0.73}{##1}}}
\def\PYG@tok@kt{\def\PYG@tc##1{\textcolor[rgb]{0.56,0.13,0.00}{##1}}}
\def\PYG@tok@sc{\def\PYG@tc##1{\textcolor[rgb]{0.25,0.44,0.63}{##1}}}
\def\PYG@tok@sb{\def\PYG@tc##1{\textcolor[rgb]{0.25,0.44,0.63}{##1}}}
\def\PYG@tok@k{\let\PYG@bf=\textbf\def\PYG@tc##1{\textcolor[rgb]{0.00,0.44,0.13}{##1}}}
\def\PYG@tok@se{\let\PYG@bf=\textbf\def\PYG@tc##1{\textcolor[rgb]{0.25,0.44,0.63}{##1}}}
\def\PYG@tok@sd{\let\PYG@it=\textit\def\PYG@tc##1{\textcolor[rgb]{0.25,0.44,0.63}{##1}}}

\def\PYGZbs{\char`\\}
\def\PYGZus{\char`\_}
\def\PYGZob{\char`\{}
\def\PYGZcb{\char`\}}
\def\PYGZca{\char`\^}
\def\PYGZsh{\char`\#}
\def\PYGZpc{\char`\%}
\def\PYGZdl{\char`\$}
\def\PYGZti{\char`\~}
% for compatibility with earlier versions
\def\PYGZat{@}
\def\PYGZlb{[}
\def\PYGZrb{]}
\makeatother

\begin{document}

\maketitle
\tableofcontents
\phantomsection\label{index::doc}


PDF Version
\phantomsection\label{index:module-qikify}\index{qikify (module)}
\textbf{Date}: March 27, 2012 \textbf{Version}: 0.2.0

\textbf{Source Repository:} \href{http://github.com/trela/qikify}{http://github.com/trela/qikify}

\textbf{qikify} is a \href{http://www.python.org}{Python} package providing data
structures and algorithms for semiconductor manufacturing data analysis.

\begin{notice}{note}{Note:}
This documentation assumes general familiarity with NumPy. If you haven't
used NumPy much or at all, do invest some time in \href{http://docs.scipy.org}{learning about NumPy} first.
\end{notice}

See the package overview for more detail about what's in the library.


\chapter{API Reference}
\label{api:api-reference}\label{api:api}\label{api::doc}\label{api:qikify-machine-learning}

\section{General functions}
\label{api:api-functions}\label{api:general-functions}

\subsection{Helper Functions}
\label{api:helper-functions}
\begin{longtable}{ll}
\hline
\endfirsthead

\multicolumn{2}{c}%
{{\bfseries \tablename\ \thetable{} -- continued from previous page}} \\
\hline
\endhead

\hline \multicolumn{2}{|r|}{{Continued on next page}} \\ \hline
\endfoot

\hline
\endlastfoot


{\hyperref[generated/qikify.helpers.create_logger:qikify.helpers.create_logger]{\code{create\_logger}}}(logmodule)
 & 

\\\hline

{\hyperref[generated/qikify.helpers.bool2symmetric:qikify.helpers.bool2symmetric]{\code{bool2symmetric}}}(data)
 & 
Changes True/False data to +1/-1 symmetric.
\\\hline

{\hyperref[generated/qikify.helpers.standardize:qikify.helpers.standardize]{\code{standardize}}}(X{[}, scaleDict, reverse{]})
 & 
Facilitates standardizing data by subtracting the mean and dividing by the standard deviation.
\\\hline

{\hyperref[generated/qikify.helpers.zeroMatrixDiagonal:qikify.helpers.zeroMatrixDiagonal]{\code{zeroMatrixDiagonal}}}(X)
 & 
Set the diagonal of a matrix to all zeros.
\\\hline

{\hyperref[generated/qikify.helpers.getParetoFront:qikify.helpers.getParetoFront]{\code{getParetoFront}}}(data)
 & 
Extracts the 2D Pareto-optimal front from a 2D numpy array.
\\\hline

{\hyperref[generated/qikify.helpers.is1D:qikify.helpers.is1D]{\code{is1D}}}(data)
 & 
Determine if data is 1-dimensional.
\\\hline

{\hyperref[generated/qikify.helpers.partition:qikify.helpers.partition]{\code{partition}}}(data{[}, threshold, verbose{]})
 & 
Partitions data into training and test sets.
\\\hline

{\hyperref[generated/qikify.helpers.nmse:qikify.helpers.nmse]{\code{nmse}}}(yhat, y{[}, min\_y, max\_y{]})
 & 
Calculates the normalized mean-squared error.
\\\hline

{\hyperref[generated/qikify.helpers.computeR2:qikify.helpers.computeR2]{\code{computeR2}}}(yhat, y)
 & 
Computes R-squared coefficient of determination.
\\\hline
\end{longtable}



\subsubsection{qikify.helpers.create\_logger}
\label{generated/qikify.helpers.create_logger::doc}\label{generated/qikify.helpers.create_logger:qikify-helpers-create-logger}\index{create\_logger (in module qikify.helpers)}

\begin{fulllineitems}
\phantomsection\label{generated/qikify.helpers.create_logger:qikify.helpers.create_logger}\pysigline{\code{qikify.helpers.}\bfcode{create\_logger}\strong{ = \textless{}function create\_logger at 0x1037a1e60\textgreater{}}}
\end{fulllineitems}



\subsubsection{qikify.helpers.bool2symmetric}
\label{generated/qikify.helpers.bool2symmetric:qikify-helpers-bool2symmetric}\label{generated/qikify.helpers.bool2symmetric::doc}\index{bool2symmetric (in module qikify.helpers)}

\begin{fulllineitems}
\phantomsection\label{generated/qikify.helpers.bool2symmetric:qikify.helpers.bool2symmetric}\pysigline{\code{qikify.helpers.}\bfcode{bool2symmetric}\strong{ = \textless{}function bool2symmetric at 0x1037a8488\textgreater{}}}
Changes True/False data to +1/-1 symmetric.

\end{fulllineitems}



\subsubsection{qikify.helpers.standardize}
\label{generated/qikify.helpers.standardize:qikify-helpers-standardize}\label{generated/qikify.helpers.standardize::doc}\index{standardize (in module qikify.helpers)}

\begin{fulllineitems}
\phantomsection\label{generated/qikify.helpers.standardize:qikify.helpers.standardize}\pysigline{\code{qikify.helpers.}\bfcode{standardize}\strong{ = \textless{}function standardize at 0x1037a8500\textgreater{}}}
Facilitates standardizing data by subtracting the mean and dividing by
the standard deviation. Set reverse to True to perform the inverse 
operation.
\begin{quote}\begin{description}
\item[{Parameters }] \leavevmode
\textbf{X} : numpy ndarray, or pandas.DataFrame
\begin{quote}

Data for which we want pareto-optimal front.
\end{quote}

\textbf{scaleDict: dict, default None} :
\begin{quote}

Dictionary with elements mean/std to control standardization.
\end{quote}

\textbf{reverse: boolean, default False} :
\begin{quote}

If this flag is set, the standardization will be reversed; e.g.,
we take a dataset with zero mean and unit variance and change to
dataset with mean=scaleDict.mean and std=scaleDict.std.
\end{quote}

\end{description}\end{quote}
\paragraph{Examples}

TODO

\end{fulllineitems}



\subsubsection{qikify.helpers.zeroMatrixDiagonal}
\label{generated/qikify.helpers.zeroMatrixDiagonal:qikify-helpers-zeromatrixdiagonal}\label{generated/qikify.helpers.zeroMatrixDiagonal::doc}\index{zeroMatrixDiagonal (in module qikify.helpers)}

\begin{fulllineitems}
\phantomsection\label{generated/qikify.helpers.zeroMatrixDiagonal:qikify.helpers.zeroMatrixDiagonal}\pysigline{\code{qikify.helpers.}\bfcode{zeroMatrixDiagonal}\strong{ = \textless{}function zeroMatrixDiagonal at 0x1037a8578\textgreater{}}}
Set the diagonal of a matrix to all zeros.
\begin{quote}\begin{description}
\item[{Parameters }] \leavevmode
\textbf{X} : numpy ndarray
\begin{quote}

Matrix on which to zero out the diagonal.
\end{quote}

\end{description}\end{quote}
\paragraph{Examples}

Xp = zeroMatrixDiagonal(X)

\end{fulllineitems}



\subsubsection{qikify.helpers.getParetoFront}
\label{generated/qikify.helpers.getParetoFront::doc}\label{generated/qikify.helpers.getParetoFront:qikify-helpers-getparetofront}\index{getParetoFront (in module qikify.helpers)}

\begin{fulllineitems}
\phantomsection\label{generated/qikify.helpers.getParetoFront:qikify.helpers.getParetoFront}\pysigline{\code{qikify.helpers.}\bfcode{getParetoFront}\strong{ = \textless{}function getParetoFront at 0x1037a85f0\textgreater{}}}
Extracts the 2D Pareto-optimal front from a 2D numpy array.
\begin{quote}\begin{description}
\item[{Parameters }] \leavevmode
\textbf{data} : numpy ndarray, or pandas.DataFrame
\begin{quote}

Data for which we want pareto-optimal front.
\end{quote}

\end{description}\end{quote}
\paragraph{Examples}

p = getParetoFront(data)

\end{fulllineitems}



\subsubsection{qikify.helpers.is1D}
\label{generated/qikify.helpers.is1D:qikify-helpers-is1d}\label{generated/qikify.helpers.is1D::doc}\index{is1D (in module qikify.helpers)}

\begin{fulllineitems}
\phantomsection\label{generated/qikify.helpers.is1D:qikify.helpers.is1D}\pysigline{\code{qikify.helpers.}\bfcode{is1D}\strong{ = \textless{}function is1D at 0x1037a8668\textgreater{}}}
Determine if data is 1-dimensional.

\end{fulllineitems}



\subsubsection{qikify.helpers.partition}
\label{generated/qikify.helpers.partition:qikify-helpers-partition}\label{generated/qikify.helpers.partition::doc}\index{partition (in module qikify.helpers)}

\begin{fulllineitems}
\phantomsection\label{generated/qikify.helpers.partition:qikify.helpers.partition}\pysigline{\code{qikify.helpers.}\bfcode{partition}\strong{ = \textless{}function partition at 0x1037a86e0\textgreater{}}}
Partitions data into training and test sets. Assumes the last column of
data is y.
\begin{quote}\begin{description}
\item[{Parameters }] \leavevmode
\textbf{data} : numpy ndarray, or pandas.DataFrame
\begin{quote}

Data to partition into training and test sets.
\end{quote}

\textbf{threshold} : float
\begin{quote}

Determines ratio of training : test.
\end{quote}

\end{description}\end{quote}
\paragraph{Examples}

TODO

\end{fulllineitems}



\subsubsection{qikify.helpers.nmse}
\label{generated/qikify.helpers.nmse::doc}\label{generated/qikify.helpers.nmse:qikify-helpers-nmse}\index{nmse (in module qikify.helpers)}

\begin{fulllineitems}
\phantomsection\label{generated/qikify.helpers.nmse:qikify.helpers.nmse}\pysigline{\code{qikify.helpers.}\bfcode{nmse}\strong{ = \textless{}function nmse at 0x1037a8758\textgreater{}}}
Calculates the normalized mean-squared error.
\begin{quote}\begin{description}
\item[{Parameters }] \leavevmode
\textbf{yhat} : 1d array or list of floats
\begin{quote}

estimated values of y
\end{quote}

\textbf{y} : 1d array or list of floats
\begin{quote}

true values
\end{quote}

\textbf{min\_y, max\_y} : float, float
\begin{quote}

roughly the min and max; they do not have to be the perfect values of min and max, because
they're just here to scale the output into a roughly {[}0,1{]} range
\end{quote}

\end{description}\end{quote}
\paragraph{Examples}

nmse = nmse(yhat, y)

\end{fulllineitems}



\subsubsection{qikify.helpers.computeR2}
\label{generated/qikify.helpers.computeR2::doc}\label{generated/qikify.helpers.computeR2:qikify-helpers-computer2}\index{computeR2 (in module qikify.helpers)}

\begin{fulllineitems}
\phantomsection\label{generated/qikify.helpers.computeR2:qikify.helpers.computeR2}\pysigline{\code{qikify.helpers.}\bfcode{computeR2}\strong{ = \textless{}function computeR2 at 0x1037a87d0\textgreater{}}}
Computes R-squared coefficient of determination.
\begin{quote}

R2 = 1 - sum((y\_hat - y\_test)**2) / sum((y\_test - np.mean(y\_test))**2)
\end{quote}
\begin{quote}\begin{description}
\item[{Parameters }] \leavevmode
\textbf{yhat} : 1d array or list of floats -- estimated values of y

\textbf{y} : 1d array or list of floats -- true values

\end{description}\end{quote}
\paragraph{Examples}

r2 = computeR2(yhat, y)

\end{fulllineitems}



\section{Models}
\label{api:models}

\subsection{Chip}
\label{api:chip}
\begin{longtable}{ll}
\hline
\endfirsthead

\multicolumn{2}{c}%
{{\bfseries \tablename\ \thetable{} -- continued from previous page}} \\
\hline
\endhead

\hline \multicolumn{2}{|r|}{{Continued on next page}} \\ \hline
\endfoot

\hline
\endlastfoot


{\hyperref[generated/qikify.models.chip.Chip:qikify.models.chip.Chip]{\code{Chip}}}(chip\_dict{[}, LCT\_prefix{]})
 & 
Encapsulates chip-level data.
\\\hline
\end{longtable}



\subsubsection{qikify.models.chip.Chip}
\label{generated/qikify.models.chip.Chip:qikify-models-chip-chip}\label{generated/qikify.models.chip.Chip::doc}\index{Chip (in module qikify.models.chip)}

\begin{fulllineitems}
\phantomsection\label{generated/qikify.models.chip.Chip:qikify.models.chip.Chip}\pysigline{\code{qikify.models.chip.}\bfcode{Chip}\strong{ = \textless{}class `qikify.models.chip.Chip'\textgreater{}}}
Encapsulates chip-level data.

\end{fulllineitems}



\section{Views}
\label{api:views}
\begin{longtable}{ll}
\hline
\endfirsthead

\multicolumn{2}{c}%
{{\bfseries \tablename\ \thetable{} -- continued from previous page}} \\
\hline
\endhead

\hline \multicolumn{2}{|r|}{{Continued on next page}} \\ \hline
\endfoot

\hline
\endlastfoot


{\hyperref[generated/qikify.views.charts.syntheticAndReal:qikify.views.charts.syntheticAndReal]{\code{syntheticAndReal}}}(sData, bData, d1, d2, filename)
 & 

\\\hline

{\hyperref[generated/qikify.views.charts.histogram:qikify.views.charts.histogram]{\code{histogram}}}(sData, bData, i{[}, filename{]})
 & 

\\\hline

{\hyperref[generated/qikify.views.charts.yp_vs_y:qikify.views.charts.yp_vs_y]{\code{yp\_vs\_y}}}(yp, y{[}, filename{]})
 & 
This method plots y predicted vs.
\\\hline

{\hyperref[generated/qikify.views.charts.qq:qikify.views.charts.qq]{\code{qq}}}(x{[}, filename{]})
 & 

\\\hline

{\hyperref[generated/qikify.views.charts.coef_path:qikify.views.charts.coef_path]{\code{coef\_path}}}(coefs)
 & 
Plot the coefficient paths generated by elastic net / lasso.
\\\hline

{\hyperref[generated/qikify.views.charts.pairs:qikify.views.charts.pairs]{\code{pairs}}}(data{[}, labels, filename{]})
 & 
Generates something similar to R pairs()
\\\hline

{\hyperref[generated/qikify.views.charts.te_and_yl:qikify.views.charts.te_and_yl]{\code{te\_and\_yl}}}(error, errorSyn, filename, description)
 & 

\\\hline

{\hyperref[generated/qikify.views.charts.laplacianScores:qikify.views.charts.laplacianScores]{\code{laplacianScores}}}(filename, Scores, Ranking)
 & 

\\\hline

{\hyperref[generated/qikify.views.charts.wafermap:qikify.views.charts.wafermap]{\code{wafermap}}}(x, y, val{[}, filename{]})
 & 

\\\hline
\end{longtable}



\subsection{qikify.views.charts.syntheticAndReal}
\label{generated/qikify.views.charts.syntheticAndReal::doc}\label{generated/qikify.views.charts.syntheticAndReal:qikify-views-charts-syntheticandreal}\index{syntheticAndReal (in module qikify.views.charts)}

\begin{fulllineitems}
\phantomsection\label{generated/qikify.views.charts.syntheticAndReal:qikify.views.charts.syntheticAndReal}\pysigline{\code{qikify.views.charts.}\bfcode{syntheticAndReal}\strong{ = \textless{}function syntheticAndReal at 0x1059d65f0\textgreater{}}}
\end{fulllineitems}



\subsection{qikify.views.charts.histogram}
\label{generated/qikify.views.charts.histogram::doc}\label{generated/qikify.views.charts.histogram:qikify-views-charts-histogram}\index{histogram (in module qikify.views.charts)}

\begin{fulllineitems}
\phantomsection\label{generated/qikify.views.charts.histogram:qikify.views.charts.histogram}\pysigline{\code{qikify.views.charts.}\bfcode{histogram}\strong{ = \textless{}function histogram at 0x1059d6668\textgreater{}}}
\end{fulllineitems}



\subsection{qikify.views.charts.yp\_vs\_y}
\label{generated/qikify.views.charts.yp_vs_y:qikify-views-charts-yp-vs-y}\label{generated/qikify.views.charts.yp_vs_y::doc}\index{yp\_vs\_y (in module qikify.views.charts)}

\begin{fulllineitems}
\phantomsection\label{generated/qikify.views.charts.yp_vs_y:qikify.views.charts.yp_vs_y}\pysigline{\code{qikify.views.charts.}\bfcode{yp\_vs\_y}\strong{ = \textless{}function yp\_vs\_y at 0x1059d66e0\textgreater{}}}
This method plots y predicted vs. y actual on a 45-degree chart.

\end{fulllineitems}



\subsection{qikify.views.charts.qq}
\label{generated/qikify.views.charts.qq:qikify-views-charts-qq}\label{generated/qikify.views.charts.qq::doc}\index{qq (in module qikify.views.charts)}

\begin{fulllineitems}
\phantomsection\label{generated/qikify.views.charts.qq:qikify.views.charts.qq}\pysigline{\code{qikify.views.charts.}\bfcode{qq}\strong{ = \textless{}function qq at 0x1059d6758\textgreater{}}}
\end{fulllineitems}



\subsection{qikify.views.charts.coef\_path}
\label{generated/qikify.views.charts.coef_path::doc}\label{generated/qikify.views.charts.coef_path:qikify-views-charts-coef-path}\index{coef\_path (in module qikify.views.charts)}

\begin{fulllineitems}
\phantomsection\label{generated/qikify.views.charts.coef_path:qikify.views.charts.coef_path}\pysigline{\code{qikify.views.charts.}\bfcode{coef\_path}\strong{ = \textless{}function coef\_path at 0x1059d67d0\textgreater{}}}
Plot the coefficient paths generated by elastic net / lasso.

\end{fulllineitems}



\subsection{qikify.views.charts.pairs}
\label{generated/qikify.views.charts.pairs:qikify-views-charts-pairs}\label{generated/qikify.views.charts.pairs::doc}\index{pairs (in module qikify.views.charts)}

\begin{fulllineitems}
\phantomsection\label{generated/qikify.views.charts.pairs:qikify.views.charts.pairs}\pysigline{\code{qikify.views.charts.}\bfcode{pairs}\strong{ = \textless{}function pairs at 0x1059d6848\textgreater{}}}
Generates something similar to R pairs()

\end{fulllineitems}



\subsection{qikify.views.charts.te\_and\_yl}
\label{generated/qikify.views.charts.te_and_yl:qikify-views-charts-te-and-yl}\label{generated/qikify.views.charts.te_and_yl::doc}\index{te\_and\_yl (in module qikify.views.charts)}

\begin{fulllineitems}
\phantomsection\label{generated/qikify.views.charts.te_and_yl:qikify.views.charts.te_and_yl}\pysigline{\code{qikify.views.charts.}\bfcode{te\_and\_yl}\strong{ = \textless{}function te\_and\_yl at 0x1059d68c0\textgreater{}}}
\end{fulllineitems}



\subsection{qikify.views.charts.laplacianScores}
\label{generated/qikify.views.charts.laplacianScores:qikify-views-charts-laplacianscores}\label{generated/qikify.views.charts.laplacianScores::doc}\index{laplacianScores (in module qikify.views.charts)}

\begin{fulllineitems}
\phantomsection\label{generated/qikify.views.charts.laplacianScores:qikify.views.charts.laplacianScores}\pysigline{\code{qikify.views.charts.}\bfcode{laplacianScores}\strong{ = \textless{}function laplacianScores at 0x1059d6938\textgreater{}}}
\end{fulllineitems}



\subsection{qikify.views.charts.wafermap}
\label{generated/qikify.views.charts.wafermap:qikify-views-charts-wafermap}\label{generated/qikify.views.charts.wafermap::doc}\index{wafermap (in module qikify.views.charts)}

\begin{fulllineitems}
\phantomsection\label{generated/qikify.views.charts.wafermap:qikify.views.charts.wafermap}\pysigline{\code{qikify.views.charts.}\bfcode{wafermap}\strong{ = \textless{}function wafermap at 0x1059d69b0\textgreater{}}}
\end{fulllineitems}



\section{Controllers}
\label{api:controllers}
\begin{longtable}{ll}
\hline
\endfirsthead

\multicolumn{2}{c}%
{{\bfseries \tablename\ \thetable{} -- continued from previous page}} \\
\hline
\endhead

\hline \multicolumn{2}{|r|}{{Continued on next page}} \\ \hline
\endfoot

\hline
\endlastfoot


{\hyperref[generated/qikify.controllers.identifyOutliers.identifyOutliers:qikify.controllers.identifyOutliers.identifyOutliers]{\code{identifyOutliers.identifyOutliers}}}(data{[}, k{]})
 & 
Compare a dataset against mu +/- k*sigma limits, and
\\\hline

{\hyperref[generated/qikify.controllers.identifyOutliers.identifyOutliersSpecs:qikify.controllers.identifyOutliers.identifyOutliersSpecs]{\code{identifyOutliers.identifyOutliersSpecs}}}(data, ...)
 & 
Compare a dataset against expanded spec limits, and
\\\hline
\end{longtable}



\subsection{qikify.controllers.identifyOutliers.identifyOutliers}
\label{generated/qikify.controllers.identifyOutliers.identifyOutliers:qikify-controllers-identifyoutliers-identifyoutliers}\label{generated/qikify.controllers.identifyOutliers.identifyOutliers::doc}\index{identifyOutliers (in module qikify.controllers.identifyOutliers)}

\begin{fulllineitems}
\phantomsection\label{generated/qikify.controllers.identifyOutliers.identifyOutliers:qikify.controllers.identifyOutliers.identifyOutliers}\pysigline{\code{qikify.controllers.identifyOutliers.}\bfcode{identifyOutliers}\strong{ = \textless{}function identifyOutliers at 0x102a792a8\textgreater{}}}
Compare a dataset against mu +/- k*sigma limits, and
return a boolean vector with False elements denoting outliers.
\begin{quote}\begin{description}
\item[{Parameters }] \leavevmode
\textbf{data} : Contains data stored in a pandas DataFrame or Series.

\end{description}\end{quote}

\end{fulllineitems}



\subsection{qikify.controllers.identifyOutliers.identifyOutliersSpecs}
\label{generated/qikify.controllers.identifyOutliers.identifyOutliersSpecs:qikify-controllers-identifyoutliers-identifyoutliersspecs}\label{generated/qikify.controllers.identifyOutliers.identifyOutliersSpecs::doc}\index{identifyOutliersSpecs (in module qikify.controllers.identifyOutliers)}

\begin{fulllineitems}
\phantomsection\label{generated/qikify.controllers.identifyOutliers.identifyOutliersSpecs:qikify.controllers.identifyOutliers.identifyOutliersSpecs}\pysigline{\code{qikify.controllers.identifyOutliers.}\bfcode{identifyOutliersSpecs}\strong{ = \textless{}function identifyOutliersSpecs at 0x102a79320\textgreater{}}}
Compare a dataset against expanded spec limits, and
return a boolean vector with False elements denoting outliers.
\begin{quote}\begin{description}
\item[{Parameters }] \leavevmode
\textbf{data} : Contains data stored in a pandas DataFrame or Series.

\end{description}\end{quote}

\end{fulllineitems}



\subsection{KDE}
\label{api:kde}
\begin{longtable}{ll}
\hline
\endfirsthead

\multicolumn{2}{c}%
{{\bfseries \tablename\ \thetable{} -- continued from previous page}} \\
\hline
\endhead

\hline \multicolumn{2}{|r|}{{Continued on next page}} \\ \hline
\endfoot

\hline
\endlastfoot


{\hyperref[generated/qikify.controllers.KDE.KDE:qikify.controllers.KDE.KDE]{\code{KDE}}}()
 & 

\\\hline

\code{KDE.\_\_init\_\_}()
 & 
Performs non-parametric kernel density estimation.
\\\hline

\code{KDE.run}(X{[}, specs, nSamples, counts, a, bounds{]})
 & 
Primary execution point.
\\\hline
\end{longtable}



\subsubsection{qikify.controllers.KDE.KDE}
\label{generated/qikify.controllers.KDE.KDE:qikify-controllers-kde-kde}\label{generated/qikify.controllers.KDE.KDE::doc}\index{KDE (in module qikify.controllers.KDE)}

\begin{fulllineitems}
\phantomsection\label{generated/qikify.controllers.KDE.KDE:qikify.controllers.KDE.KDE}\pysigline{\code{qikify.controllers.KDE.}\bfcode{KDE}\strong{ = \textless{}class `qikify.controllers.KDE.KDE'\textgreater{}}}
\end{fulllineitems}



\subsubsection{qikify.controllers.KDE.KDE.\_\_init\_\_}
\label{generated/qikify.controllers.KDE.KDE.__init__::doc}\label{generated/qikify.controllers.KDE.KDE.__init__:qikify-controllers-kde-kde-init}

\subsubsection{qikify.controllers.KDE.KDE.run}
\label{generated/qikify.controllers.KDE.KDE.run::doc}\label{generated/qikify.controllers.KDE.KDE.run:qikify-controllers-kde-kde-run}

\subsection{Recipes}
\label{api:recipes}
\begin{longtable}{ll}
\hline
\endfirsthead

\multicolumn{2}{c}%
{{\bfseries \tablename\ \thetable{} -- continued from previous page}} \\
\hline
\endhead

\hline \multicolumn{2}{|r|}{{Continued on next page}} \\ \hline
\endfoot

\hline
\endlastfoot


{\hyperref[generated/qikify.recipes.atesim:module-qikify.recipes.atesim]{\code{atesim}}}
 & 

\\\hline

{\hyperref[generated/qikify.recipes.basic_ML_testing:module-qikify.recipes.basic_ML_testing]{\code{basic\_ML\_testing}}}
 & 

\\\hline

{\hyperref[generated/qikify.recipes.two_tier_test:module-qikify.recipes.two_tier_test]{\code{two\_tier\_test}}}
 & 

\\\hline
\end{longtable}



\subsubsection{qikify.recipes.atesim}
\label{generated/qikify.recipes.atesim:qikify-recipes-atesim}\label{generated/qikify.recipes.atesim::doc}\label{generated/qikify.recipes.atesim:module-qikify.recipes.atesim}\index{qikify.recipes.atesim (module)}

\subsubsection{qikify.recipes.basic\_ML\_testing}
\label{generated/qikify.recipes.basic_ML_testing:qikify-recipes-basic-ml-testing}\label{generated/qikify.recipes.basic_ML_testing:module-qikify.recipes.basic_ML_testing}\label{generated/qikify.recipes.basic_ML_testing::doc}\index{qikify.recipes.basic\_ML\_testing (module)}

\subsubsection{qikify.recipes.two\_tier\_test}
\label{generated/qikify.recipes.two_tier_test:qikify-recipes-two-tier-test}\label{generated/qikify.recipes.two_tier_test:module-qikify.recipes.two_tier_test}\label{generated/qikify.recipes.two_tier_test::doc}\index{qikify.recipes.two\_tier\_test (module)}

\renewcommand{\indexname}{Python Module Index}
\begin{theindex}
\def\bigletter#1{{\Large\sffamily#1}\nopagebreak\vspace{1mm}}
\bigletter{q}
\item {\texttt{qikify}}, \pageref{index:module-qikify}
\item {\texttt{qikify.recipes.atesim}}, \pageref{generated/qikify.recipes.atesim:module-qikify.recipes.atesim}
\item {\texttt{qikify.recipes.basic\_ML\_testing}}, \pageref{generated/qikify.recipes.basic_ML_testing:module-qikify.recipes.basic_ML_testing}
\item {\texttt{qikify.recipes.two\_tier\_test}}, \pageref{generated/qikify.recipes.two_tier_test:module-qikify.recipes.two_tier_test}
\end{theindex}

\renewcommand{\indexname}{Index}
\printindex
\end{document}
